% Options for packages loaded elsewhere
% Options for packages loaded elsewhere
\PassOptionsToPackage{unicode}{hyperref}
\PassOptionsToPackage{hyphens}{url}
\PassOptionsToPackage{dvipsnames,svgnames,x11names}{xcolor}
%
\documentclass[
  letterpaper,
  DIV=11,
  numbers=noendperiod]{scrartcl}
\usepackage{xcolor}
\usepackage{amsmath,amssymb}
\setcounter{secnumdepth}{5}
\usepackage{iftex}
\ifPDFTeX
  \usepackage[T1]{fontenc}
  \usepackage[utf8]{inputenc}
  \usepackage{textcomp} % provide euro and other symbols
\else % if luatex or xetex
  \usepackage{unicode-math} % this also loads fontspec
  \defaultfontfeatures{Scale=MatchLowercase}
  \defaultfontfeatures[\rmfamily]{Ligatures=TeX,Scale=1}
\fi
\usepackage{lmodern}
\ifPDFTeX\else
  % xetex/luatex font selection
\fi
% Use upquote if available, for straight quotes in verbatim environments
\IfFileExists{upquote.sty}{\usepackage{upquote}}{}
\IfFileExists{microtype.sty}{% use microtype if available
  \usepackage[]{microtype}
  \UseMicrotypeSet[protrusion]{basicmath} % disable protrusion for tt fonts
}{}
\makeatletter
\@ifundefined{KOMAClassName}{% if non-KOMA class
  \IfFileExists{parskip.sty}{%
    \usepackage{parskip}
  }{% else
    \setlength{\parindent}{0pt}
    \setlength{\parskip}{6pt plus 2pt minus 1pt}}
}{% if KOMA class
  \KOMAoptions{parskip=half}}
\makeatother
% Make \paragraph and \subparagraph free-standing
\makeatletter
\ifx\paragraph\undefined\else
  \let\oldparagraph\paragraph
  \renewcommand{\paragraph}{
    \@ifstar
      \xxxParagraphStar
      \xxxParagraphNoStar
  }
  \newcommand{\xxxParagraphStar}[1]{\oldparagraph*{#1}\mbox{}}
  \newcommand{\xxxParagraphNoStar}[1]{\oldparagraph{#1}\mbox{}}
\fi
\ifx\subparagraph\undefined\else
  \let\oldsubparagraph\subparagraph
  \renewcommand{\subparagraph}{
    \@ifstar
      \xxxSubParagraphStar
      \xxxSubParagraphNoStar
  }
  \newcommand{\xxxSubParagraphStar}[1]{\oldsubparagraph*{#1}\mbox{}}
  \newcommand{\xxxSubParagraphNoStar}[1]{\oldsubparagraph{#1}\mbox{}}
\fi
\makeatother


\usepackage{longtable,booktabs,array}
\usepackage{calc} % for calculating minipage widths
% Correct order of tables after \paragraph or \subparagraph
\usepackage{etoolbox}
\makeatletter
\patchcmd\longtable{\par}{\if@noskipsec\mbox{}\fi\par}{}{}
\makeatother
% Allow footnotes in longtable head/foot
\IfFileExists{footnotehyper.sty}{\usepackage{footnotehyper}}{\usepackage{footnote}}
\makesavenoteenv{longtable}
\usepackage{graphicx}
\makeatletter
\newsavebox\pandoc@box
\newcommand*\pandocbounded[1]{% scales image to fit in text height/width
  \sbox\pandoc@box{#1}%
  \Gscale@div\@tempa{\textheight}{\dimexpr\ht\pandoc@box+\dp\pandoc@box\relax}%
  \Gscale@div\@tempb{\linewidth}{\wd\pandoc@box}%
  \ifdim\@tempb\p@<\@tempa\p@\let\@tempa\@tempb\fi% select the smaller of both
  \ifdim\@tempa\p@<\p@\scalebox{\@tempa}{\usebox\pandoc@box}%
  \else\usebox{\pandoc@box}%
  \fi%
}
% Set default figure placement to htbp
\def\fps@figure{htbp}
\makeatother





\setlength{\emergencystretch}{3em} % prevent overfull lines

\providecommand{\tightlist}{%
  \setlength{\itemsep}{0pt}\setlength{\parskip}{0pt}}



 


\KOMAoption{captions}{tableheading}
\makeatletter
\@ifpackageloaded{caption}{}{\usepackage{caption}}
\AtBeginDocument{%
\ifdefined\contentsname
  \renewcommand*\contentsname{Table of contents}
\else
  \newcommand\contentsname{Table of contents}
\fi
\ifdefined\listfigurename
  \renewcommand*\listfigurename{List of Figures}
\else
  \newcommand\listfigurename{List of Figures}
\fi
\ifdefined\listtablename
  \renewcommand*\listtablename{List of Tables}
\else
  \newcommand\listtablename{List of Tables}
\fi
\ifdefined\figurename
  \renewcommand*\figurename{Figure}
\else
  \newcommand\figurename{Figure}
\fi
\ifdefined\tablename
  \renewcommand*\tablename{Table}
\else
  \newcommand\tablename{Table}
\fi
}
\@ifpackageloaded{float}{}{\usepackage{float}}
\floatstyle{ruled}
\@ifundefined{c@chapter}{\newfloat{codelisting}{h}{lop}}{\newfloat{codelisting}{h}{lop}[chapter]}
\floatname{codelisting}{Listing}
\newcommand*\listoflistings{\listof{codelisting}{List of Listings}}
\makeatother
\makeatletter
\makeatother
\makeatletter
\@ifpackageloaded{caption}{}{\usepackage{caption}}
\@ifpackageloaded{subcaption}{}{\usepackage{subcaption}}
\makeatother
\usepackage{bookmark}
\IfFileExists{xurl.sty}{\usepackage{xurl}}{} % add URL line breaks if available
\urlstyle{same}
\hypersetup{
  pdftitle={Technical Write-up},
  colorlinks=true,
  linkcolor={blue},
  filecolor={Maroon},
  citecolor={Blue},
  urlcolor={Blue},
  pdfcreator={LaTeX via pandoc}}


\title{Technical Write-up}
\author{}
\date{}
\begin{document}
\maketitle

\renewcommand*\contentsname{Table of contents}
{
\hypersetup{linkcolor=}
\setcounter{tocdepth}{3}
\tableofcontents
}

Abstract

This project looks at how drought conditions relate to farm income
across U.S. states using state data from 1949 until 2014. Drought
severity is measured with the Palmer Drought Severity Index (PDSI), and
farm income is measured using state-level income totals categorized by
farming product. The analysis follows a consistent workflow to clean,
combine, and analyze the data. When income is analyzed at the national
level, the relationship with drought appears weak. After adjusting
income within each state, a much clearer relationship appears. This
shows that changes in drought conditions are linked to changes in farm
income within states, even if this pattern is hidden in overall
comparisons.

\begin{enumerate}
\def\labelenumi{\arabic{enumi}.}
\tightlist
\item
  Introduction
\end{enumerate}

Studying how weather affects farm income is challenging because U.S.
states differ widely from one another. Some states consistently earn
much more farm income than others due to differences in crops, land use,
and scale of production. When all states are analyzed together, these
baseline differences can dominate smaller year-to-year changes.

The goal of this project is to determine whether drought conditions are
related to changes in farm income once these baseline differences are
removed. Rather than focusing on raw income levels, the analysis
examines how income in a given year compares to a state's typical
income. This makes it easier to compare patterns across states and focus
on changes rather than overall size.

\begin{enumerate}
\def\labelenumi{\arabic{enumi}.}
\setcounter{enumi}{1}
\tightlist
\item
  Data Sources 2.1 Climate Data (PDSI)
\end{enumerate}

Monthly Palmer Drought Severity Index data were downloaded from NOAA.
The raw data are provided as fixed-width text files, so the column
locations must be specified manually in order to read them correctly.
Each observation includes a state, year, month, and a PDSI value that
reflects drought conditions.

\begin{figure}[H]

{\centering \pandocbounded{\includegraphics[keepaspectratio]{docs/tutorial_files/figure-html/cell-7-output-1.png}}

}

\caption{Raw monthly PDSI values by state and year}

\end{figure}%

2.2 Farm Income Data

Farm income data were collected from public U.S. agricultural economic
sources. These data are reported at the state-by-year level and
represent total farm income before any adjustments. The structure of
these data matches the aggregated PDSI data, which allows the two
datasets to be combined cleanly.

\begin{enumerate}
\def\labelenumi{\arabic{enumi}.}
\setcounter{enumi}{2}
\tightlist
\item
  Data Processing and Combination 3.1 Reading and Aggregating PDSI Data
\end{enumerate}

The raw PDSI text files were read using a custom function designed for
fixed-width data. This step converts the text files into tables that can
be used in Python. Monthly PDSI values were then averaged within each
state and year to create a single yearly drought value, referred to as
yearly\_avg.

3.2 Combining Climate and Income Data

The yearly PDSI data and farm income data were combined using state and
year as matching keys. The result is a state-by-year dataset where each
row represents one state in one year, with both drought and income
information.

\begin{enumerate}
\def\labelenumi{\arabic{enumi}.}
\setcounter{enumi}{3}
\tightlist
\item
  Exploratory Analysis and Data Cleaning
\end{enumerate}

Initial plots revealed a clear issue in the PDSI data for the year 2014.
In that year, all states showed unusually low values that did not follow
the patterns seen in nearby years. Because this behavior is most likely
due to a data issue rather than actual drought conditions, these extreme
values were removed before continuing the analysis.

Only values that were clearly unrealistic were filtered out. This helps
prevent these points from having too much influence on correlations and
fitted relationships.

\begin{figure}[H]

{\centering \pandocbounded{\includegraphics[keepaspectratio]{docs/tutorial_files/figure-html/cell-9-output-1.png}}

}

\caption{2014 anomaly}

\end{figure}%

\begin{enumerate}
\def\labelenumi{\arabic{enumi}.}
\setcounter{enumi}{4}
\tightlist
\item
  Methods 5.1 Correlation Analysis
\end{enumerate}

The relationship between drought conditions and farm income was first
explored using Pearson correlation values. Scatter plots were also used
to visually examine the relationship and check for general patterns or
unusual spread.

5.2 Adjusting Income Within States

To remove large income differences across states, farm income was
adjusted within each state. For each state, the average income across
all years was calculated and subtracted from each yearly value: \[ 
\text{adjusted income}_{it}
= \text{income}_{it} - \bar{\text{income}}_{i}
\]

This shifts the focus away from absolute income levels and toward how
much a given year is above or below a state's typical income.

5.3 State-Level Plots

Scatter plots colored by state were used to see whether states behave
differently. These plots make it easier to compare how income responds
to drought across states while still showing the overall trend.

\begin{enumerate}
\def\labelenumi{\arabic{enumi}.}
\setcounter{enumi}{5}
\tightlist
\item
  Results 6.1 Results Using Unadjusted Income
\end{enumerate}

When farm income is analyzed in its original form, the relationship with
PDSI appears weak. The scatter plot shows a wide range of income values,
largely driven by differences between states rather than changes over
time.

\begin{figure}[H]

{\centering \pandocbounded{\includegraphics[keepaspectratio]{tutorial_files/figure-html/cell-10-output-1.png}}

}

\caption{PDSI versus unadjusted farm income}

\end{figure}%

6.2 Results Using Adjusted Income

After adjusting income within states, the relationship between PDSI and
income becomes much clearer. The scatter plot is more tightly grouped,
and the correlation increases substantially. This suggests that when
drought conditions improve or worsen within a state, farm income tends
to change in the same direction.

When points are colored by state, most states show similar upward
trends, even though their average income levels differ. This suggests
that drought affects farm income in a similar way across states, but
around different baseline income levels.

\begin{figure}[H]

{\centering \pandocbounded{\includegraphics[keepaspectratio]{tutorial_files/figure-html/cell-18-output-1.png}}

}

\caption{PDSI versus adjusted farm income}

\end{figure}%

\begin{enumerate}
\def\labelenumi{\arabic{enumi}.}
\setcounter{enumi}{6}
\tightlist
\item
  Python Package Implementation
\end{enumerate}

All steps of the analysis are wrapped into a Python package so the
workflow can be rerun easily. Each step, including reading data,
averaging values, combining datasets, cleaning data issues, adjusting
income, and creating plots, is handled by a separate function. This
structure makes the code easier to follow and reuse.

The package follows standard Python project rules, including a
pyproject.toml file that lists dependencies and build settings. The
package can be installed and run in a clean environment without manual
setup.

\begin{enumerate}
\def\labelenumi{\arabic{enumi}.}
\setcounter{enumi}{7}
\tightlist
\item
  Discussion
\end{enumerate}

Once income differences across states are accounted for, drought
severity shows a clear relationship with farm income. Analyses that
ignore these differences may incorrectly suggest that drought has little
effect. This project shows how relatively simple adjustments can change
how results are interpreted.

\begin{enumerate}
\def\labelenumi{\arabic{enumi}.}
\setcounter{enumi}{8}
\tightlist
\item
  Limitations and Future Work
\end{enumerate}

This analysis focuses on patterns and correlations rather than
cause-and-effect claims. Future work could add regression models with
state fixed effects, include delayed drought effects, or use more
detailed income data.

\begin{enumerate}
\def\labelenumi{\arabic{enumi}.}
\setcounter{enumi}{9}
\tightlist
\item
  Conclusion
\end{enumerate}

This project shows that drought conditions are closely related to
changes in farm income within states. By adjusting income to remove
baseline differences, the analysis reveals patterns that are hidden when
all states are compared together. The workflow also provides a reusable
setup for future studies of climate and agriculture.




\end{document}
